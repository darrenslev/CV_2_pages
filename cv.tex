%
% cv.tex
%
% Author: Darren Slevin
%
% Based heavily on resume.tex by
% Jeremy B. R. Edberg <jedberg@gmail.com> http://www.jedberg.net
% and Padraig O Conbhui


%****************************************************************************%

%
% Basic setup
%
\documentclass[11pt]{article}

\usepackage{kpfonts}

\usepackage{fullpage}
\usepackage{setspace}

\pagestyle{empty}
\setlength{\tabcolsep}{0in}
\hyphenchar\font=-1

\usepackage{geometry}
\geometry{a4paper, twoside}
\setstretch{1}

\usepackage{booktabs}
\usepackage{topcapt}
%\raggedbottom
%\raggedright
\usepackage{tabulary}
\usepackage{hyphenat}
\usepackage{verbatim}

\newcommand{\superscript}[1]{\ensuremath{^{\textrm{#1}}}}
\newcommand{\subscript}[1]{\ensuremath{_{\textrm{#1}}}}

% dummy text
\usepackage[english]{babel}
\usepackage{blindtext}

\newenvironment{newparagraph}{\par\setlength{\rightskip}{3cm}}

\usepackage{tabto}

%\newcommand{\tab}[1]{%
%\settowidth{\tabcont}{#1}%
%\ifthenelse{\lengthtest{\tabcont < .25\linewidth}}%
%{\makebox[.25\linewidth][l]{#1}\ignorespaces}%
%{\makebox[.5\linewidth][l]{\color{red} #1}\ignorespaces}%
%}%

%*****************************************************************************%


%
% PDF setup
%
\usepackage{ifpdf}
\ifpdf
  \usepackage[pdftex]{hyperref}
\else
  \usepackage[hypertex]{hyperref}
\fi
\hypersetup{
    letterpaper,
    colorlinks,
    urlcolor=black,
    %pdfpagemode=none,
    pdftitle={Resume},
    pdfauthor={Padraig O Conbhui},
    pdfsubject={Resume},
    pdfkeywords={HPC, high performance computing, web development, physics}
}


%*****************************************************************************%


%
% Margin setup
%
\oddsidemargin -0.5in
\evensidemargin -0.5in
\textwidth 7.2in
\headheight -0.7in
\topmargin 0.0in
\textheight= 10.7in
%\footheight 1.0in


%*****************************************************************************%


%
% Style setup
%
%\setlength{\parindent}{0pt}
%\setlength{\parskip}{\baselineskip}
\usepackage{parskip}


%*****************************************************************************%


%
% Custom commands
%
\newcommand{\resumeSection}[1]{
    \par
    \vspace{\baselineskip}
    \large {\sc {#1}}
    \par
    \vspace{-0.9\baselineskip}
    \hrulefill
    \vspace{0.5\baselineskip}
    \par
}

\newenvironment{resumeSubSectionHeader}{
    \par
    \begin{tabular*}{\textwidth}{l@{\extracolsep{\fill}}r}
    \par
} {
    \end{tabular*}
    \par
}

\newenvironment{resumeSubSectionBody}{
    \par
    \vspace{-0.8\parskip}
    \begin{small}
    \par
} {
    \par
    \end{small}
    \par
}


%*****************************************************************************%


\begin{document}


%*****************************************************************************%


%
% Heading
%
\begin{center}
    { \huge \textbf \sc Darren Slevin}

\begin{tabular*}{\textwidth}{@{\extracolsep{\fill}}lcr}
    d.slevin@ed.ac.uk & & Mobile: +44 7834786828\\
    \hline\hline
\end{tabular*}
\end{center}

%Personal Webpage: www.darrenslevin.com

%*****************************************************************************%

\resumeSection{Profile}
I am currently a Postdoctoral Research Associate in Global Crop Yield Modelling. This project is a collaboration between the University of Edinburgh and Global Surface Intelligence (an Edinburgh-based data services 
company). I possess a strong meteorology, 
mathematics and computing background and decided to move into the atmospheric sciences field due to 
undertaking a Global Environmental Change module while studying for my undergraduate 
degree. My research interests include but are not limited to: land surface modelling, 
model uncertainty, using large datasets, satellite data, understanding how climate 
models work, high performance computing and teaching. 

%*****************************************************************************%
\resumeSection{Research Experience}

%
% University Of Edinburgh
%
\begin{resumeSubSectionHeader}

	\textbf{University of Edinburgh} & Edinburgh, Scotland \\
	 \emph{Postdoctoral Research Associate} & \emph{2016 to present} \\ 

\end{resumeSubSectionHeader}
\begin{resumeSubSectionBody}
	
	\textbf{Project title:} Global Crop Yield Modelling\\
	\textbf{Supervisor:} Prof. Mat Williams


	\begin{itemize}
    		\begin{newparagraph} 
		\item Generating observational datasets of global crop yield for wheat, maize and sugar cane.
		\item Processed-based modelling using the CARDAMOM framework.         
		\end{newparagraph}	        
            
    \end{itemize}
\end{resumeSubSectionBody}

%
% University Of Edinburgh
%
\begin{resumeSubSectionHeader}

	\textbf{University of Edinburgh} & Edinburgh, Scotland \\
	 \emph{Postdoctoral Research Associate} & \emph{2015 to 2016} \\ 

\end{resumeSubSectionHeader}
\begin{resumeSubSectionBody}
	
	\textbf{Project title:} Carbon Cycle Processes in Permafrost Systems (CYCLOPS)\\
	\textbf{Supervisor:} Prof. Mat Williams


	\begin{itemize}
    		\begin{newparagraph} 
		\item Carbon cycle modelling at various study sites in the Canadian Arctic for the 
		CYCLOPS project using the SPA-CYCLOPS ecosystem model.       
		\end{newparagraph}	        
            
    \end{itemize}
\end{resumeSubSectionBody}

%*****************************************************************************%
\resumeSection{Education}

%
% University Of Edinburgh
%
\begin{resumeSubSectionHeader}

    \textbf{University of Edinburgh} & Edinburgh, Scotland \\
    \emph{PhD, Atmospheric and Environmental Sciences} & \emph{2011 to 2015} \\

\end{resumeSubSectionHeader}
\begin{resumeSubSectionBody}
		
   	\textbf{Thesis title:} Investigating sources of uncertainty associated with the JULES land surface model.\\
	\textbf{Supervisors:} Prof. Simon Tett and Prof. Mat Williams

    %\begin{itemize}
    		%\begin{newparagraph} 
		%\item Evaluated the performance of the JULES land surface model at both site and global scales 
		%using a range of observation and reanalysis datasets.
		%\item Experience using global gridded meteorological datasets, such as WFDEI 
        %and PRINCETON, and satellite data products, such as MODIS LAI and GPP. 
		 
		 %\item Explored and identified sources of uncertainty in the Joint UK Land Environment 
		 %Simulator (JULES) land surface model when simulating Gross Primary Productivity (GPP) at 
		 %various scales (point, regional and global).
		 
		 %\item Viva passed on 3\superscript{rd} November 2015.        
        
        %\item Evaluated the performance of the UK Met Office land surface model, JULES, for 
        %both single-point and global simulations, using a range of observation and reanalysis datasets. 
        %\item Installed and setup the JULES model in both serial and parallel mode on the Edinburgh
        %University cluster (Edinburgh Compute and Data Facilities).
        %\item Submitted model jobs to the University cluster using Grid engine scripts (qlogin, qsub, qstat, 
        %qdel), for both serial and parallel model configurations.      
        %\item Built and installed the necessary libraries (HDF5, netCDF4, parallel-netcdf, zlib, szip) for the 
        %installation of JULES.
        %\item Modifying the JULES code, which is written in Fortran 90, in order to drive 
        %it with satellite data. 
        %\item Experience using global gridded meteorological datasets, such as WFDEI 
        %and PRINCETON, and satellite data products, such as MODIS LAI and GPP.    
		%\item Analysing and comparing model output to observations
		%\end{newparagraph}	        
            
    %\end{itemize}

\end{resumeSubSectionBody}

%
% University College, Dublin
% MSc, Meteorology
%
\begin{resumeSubSectionHeader}

    \textbf{University College Dublin} & Dublin, Ireland \\
    \emph{MSc, Meteorology} & \emph{2010 to 2011} \\
    \emph{Grade: 1.1}

\end{resumeSubSectionHeader}
\begin{resumeSubSectionBody}
	%\begin{newparagraph}
	%One year course in meteorology and climate in which topics ranging from the 
	%practical (Synoptic Meteorology) to the theoretical (Dynamic Meteorology) are studied.        
	%\end{newparagraph}

    \begin{description}
    %    \item{\bf Subjects:}
    %     	Dynamic Meteorology, Physical Meteorology, Climate Dynamics, Numerical Weather\\ 
    %     	Prediction and Synoptic Meteorology.
        %\item{\bf Field Trips:}
        %		\begin{itemize}
        	%		\item Valencia Observatory (Ireland) 
        	%		\item Dorset Field Trip (UK)			
		%	\end{itemize}	              
        \item{\bf MSc dissertation:}
			Storm modelling using the Weather Research and Forecasting (WRF) model         
    \end{description}

\end{resumeSubSectionBody}

%
% University College, Dublin
% HDip, Mathematical Science
%
\begin{resumeSubSectionHeader}

    \textbf{University College Dublin} & Dublin, Ireland \\
    \emph{HDip, Mathematical Science} & \emph{2009 to 2010} \\
    \emph{Grade: 2.1}

\end{resumeSubSectionHeader}
\begin{resumeSubSectionBody}

	%\begin{newparagraph}
	%One year course in which a range of topics in algebra and analysis were taught, in order to reach 
	%a level of mathematical knowledge equivalent to that of a 4-year honours undergraduate programme.     
	%\end{newparagraph}
    %\begin{description}
    %    \item{\bf Subjects:}
    %  		Mathematical Analysis, Calculus of Several Variables, Linear Algebra, Group Theory,\\ 
    %  		Field Theory, Ring Theory, Metric Spaces, Functional Analysis and Differential Geometry            
    %\end{description}

\end{resumeSubSectionBody}

%
% Trinity College Dublin
% BA (Mod), Computer Science
%
%\newpage
\begin{resumeSubSectionHeader}

    \textbf{Trinity College Dublin} & Dublin, Ireland \\
    \emph{BA (Mod), Computer Science} & \emph{2005 to 2009} \\
    \emph{Grade: 1.1}

\end{resumeSubSectionHeader}
\begin{resumeSubSectionBody}
	%\begin{newparagraph}
    %Four year course in which the theory and practice of computer systems are taught, which included 
    %%topics in mathematics, programming, microprocessors, digital logic, telecommunications, information 
    %management, electronics and the role of computers in society. 
	%\end{newparagraph}
    \begin{description}
    		\begin{newparagraph}	
        %\item{\bf Subjects:}
        %    Mathematics, Introduction to Programming, Introduction to Computing, 
        %    Digital Logic Design, Electrotechnology, Telecommunications, Systems Programming, 
        %    Concurrent Systems and Operating Systems, Computer Architecture, Symbolic Programming, 
        %    Software Engineering, Compiler Design, Formal Methods, Information Management, Fuzzy Logic, 
        %    Distributed Systems and Mobile Communications.
           	   
       	 \item{\bf Final Year dissertation:}
			Modelling in Z 
		 \end{newparagraph}    
    \end{description}

\end{resumeSubSectionBody}

%*****************************************************************************%
\resumeSection{Work Experience}

\begin{resumeSubSectionHeader}

    \textbf{University of Edinburgh} & Edinburgh, Scotland \\
    \emph{Tutor and Demonstrator} & \emph{2011 to 2015} \\

\end{resumeSubSectionHeader}
\begin{resumeSubSectionBody}
	%\begin{newparagraph}	
	%I tutored and demonstrated for a variety of courses (ranging from 
	%%1\superscript{st} year to MSc level) within the School of Geosciences during 
	%the course of my PhD and Postdoctoral work. Responsibilities included assisting lecturers with 
	%tutorials and practical work, leading tutorials, preparing computing tutorials, 
	%marking course work, providing feedback to students and supervising exams. 
	
	%I delivered a lecture to 
	%students on the Earth Modelling and Prediction 2 course. I designed and delivered an NCL tutorial 
	%to students on the Introduction to 3-D Climate Modelling course.  
	
	%\end{newparagraph}
	\begin{description}
		\begin{newparagraph}	
        \item{\bf Subjects taught:}
            Earth Modelling and Prediction 1, Earth Modelling and Prediction 2, Meteorology: Weather and 
            Climate, Meteorology: Atmosphere and Environment, Mathematical Methods for Geophysicists, 
            Introduction to 3-D Climate Modelling, Object Oriented Software Engineering, 
            Computational Modelling for Geosciences and Environmental Pollution.
        \end{newparagraph}	         
    \end{description}
                    
\end{resumeSubSectionBody}


\begin{resumeSubSectionHeader}

    \textbf{University College Dublin} & Dublin, Ireland \\
    \emph{Mathematics Tutor} & \emph{2010 to 2011} \\

\end{resumeSubSectionHeader}
\begin{resumeSubSectionBody}
	%\begin{newparagraph}	
	%I tutored for a number of mathematics courses (1\superscript{st} and 2\superscript{nd} year) within 
	%the School of Mathematical Sciences and the School of Business while studying for my MSc. 
	%Responsibilities included leading tutorials and marking course work. 
	%\end{newparagraph}
	\begin{description}
		\begin{newparagraph}
        \item{\bf Subjects taught:}	
			Introduction to Calculus, Mathematics for Business, Combinatorics and Number Theory, 
			Calculus, Calculus and Statistics, Matrices and Vectors, and Linear Algebra for Engineers.          
        \end{newparagraph}	
    \end{description}
                    
\end{resumeSubSectionBody}


%\begin{resumeSubSectionHeader}

%    \textbf{Trinity College Dublin} & Dublin, Ireland \\
%    \emph{User Interface Design for Natural Language Processing Application} & \emph{July-Sept. 2009} \\

%\end{resumeSubSectionHeader}
%\begin{resumeSubSectionBody}
%		\begin{newparagraph}	
%        I designed and implemented a Graphical User Interface (GUI) in Java for an Enterprise Ireland-funded Language Application.
%        \end{newparagraph}	   
                    
%\end{resumeSubSectionBody}

%*****************************************************************************%
\resumeSection{Scholarships and Awards}
%
% Awards
%
\begin{resumeSubSectionHeader}

   \textbf{Awards}

\end{resumeSubSectionHeader}
\begin{resumeSubSectionBody}

    \begin{itemize}
		\item Recipient of the Victor W. Graham Prize in 2006
			\begin{itemize}
				\item Highest mark in Mathematics in Junior Freshman Computer Science Summer exams
			\end{itemize}        
        \item Several book prizes for results in Computer Science Summer exams.
    \end{itemize}

\end{resumeSubSectionBody}

\begin{resumeSubSectionHeader}

    \textbf{Scholarships}

\end{resumeSubSectionHeader}
\begin{resumeSubSectionBody}

    \begin{itemize}
		\item School Scholarship awarded by the School of GeoSciences, University of Edinburgh.  
    \end{itemize}

\end{resumeSubSectionBody}

%*****************************************************************************%
%\newpage
\resumeSection{Skills Profile}
\begin{resumeSubSectionBody}
\begin{description}
	\begin{newparagraph}	
	\item{\bf Meteorology:}
		From my MSc, I have gained an in-depth understanding of the theoretical and practical 
		aspects of meteorology, including how climate models work. During my PhD and Postdoctoral 
		research, I have furthermore developed my skills in land surface modelling, dealing 
		with large datasets and interpreting results from model simulations.	  
    \item{\bf Mathematics:} 
		From my HDip, I acquired a broad overview of pure mathematics which allowed me to 
		understand the mathematics behind atmospheric processes while studying for my MSc. 
		Explaining mathematical concepts to students during the course of my teaching has also 
		allowed me to further develop my mathematical knowledge. 
   \item{\bf Computing:}
   		During the course of my computing degree, I acquired a detailed knowledge of computer hardware 
    		and software. Programming languages studied include Java, C, C++, Prolog and SQL. During the 
    		course of my PhD and Postdoctoral research, I have programmed in Python, NCL, Fortran, CDO/NCO 
    		and shell scripting (bash). I extensively use the Linux operating system and \LaTeX/Beamer. 
    		I also have experience of using the FCM make build system and the software versioning and 
    		revision control system SVN.    		
      	\end{newparagraph}		            
\end{description}
\end{resumeSubSectionBody}

%*****************************************************************************%
\resumeSection{Publications}
\begin{resumeSubSectionBody}
\begin{itemize}
	\begin{newparagraph}	
	\item D. Slevin, S.F.B. Tett and M. Williams, `Multi-site evaluation of the JULES land surface model using 
	global and local data', Geoscientific Model Development, 8, 295-316, doi:10.5194/gmd-8-295-2015, 2015.
	
	\item D. Slevin, S.F.B. Tett, J.-F. Exbrayat, A. A. Bloom and M. Williams, `Global Evaluation of Gross Primary Productivity in the JULES Land Surface Model', Geoscientific Model Development Discussions, doi:10.5194/gmd-2016-214, in review, 2016. 
	\end{newparagraph}
	
	%\begin{newparagraph}
	
	%\end{newparagraph}	
		
\end{itemize}
\end{resumeSubSectionBody}

%*****************************************************************************%
%\resumeSection{Academic activities}

%\begin{resumeSubSectionBody}
%\begin{itemize}
%	\begin{newparagraph}
%	\item Participation in a 3 day workshop on `Understanding Uncertainty in Environmental 
%	Modelling' at the Centre for the Analysis of Time Series, London School of 
%	Economics (January 2015)
%	\item Participation in and poster presentation at the 13\superscript{th} International Swiss 
%	Climate Summer School organised by the University of Bern (Grindelwald, Switzerland, September 2014) 
%	\item Poster presentation at the JULES Summer meeting (Leicester University, July 2014)  
%	\item Participation in a 2 day ESA (European Space Agency) sponsored Land Data Assimilation 
%	workshop (University College London, March 2014)     
%	\item Oral presentation for the Contemporary Climate Research Group (Edinburgh University, December 2013)
%	\item Oral presentation at the JULES Summer meeting (Edinburgh University, June 2013)
%	\item Participation in a 1 day teaching course organised by the Higher Education 
%	Academy (HEA) (Edinburgh, March 2013)  
%%	\item Poster presentation and Session Chair at the Geosciences Annual Gradschool 
%	Conference (Edinburgh University, January 2013)
%	\item Participation in the JULES Winter meeting (Exeter University, December 2012)  
%	\item Participation in the NCEO/CEOI Joint Science Conference (Nottingham University, September 2012)	
%	\item Oral presentation at the Geosciences Postgraduate Research Conference (Edinburgh 
%	University, March 2012)
%	\item Oral presentation at the Geosciences Annual Gradschool Conference (Edinburgh University, January 2012)
%	\item Participation in the JULES Winter meeting (Reading University, January 2012)
%	\end{newparagraph}	   
%\end{itemize}
%\end{resumeSubSectionBody}	



%*****************************************************************************%
%\newpage
%\resumeSection{Other activities of interest}

%\begin{resumeSubSectionBody}	
%\begin{itemize}
%	\begin{newparagraph}	
%	\item During my PhD, I organised the weekly climate reading group for PhD students and 
%	Postdocs of Profs Simon Tett and Gabi Hegerl.
%	\item I was the Communications Officer for the Geosciences Graduate School Committee in the 
%	second year of my PhD.	
%	\item I am currently studying Arabic as part of the Languages for All programme at 
%	Edinburgh University (This is my 4\superscript{th} year).
%	\item In my free time, I enjoy reading, hiking, camping, going to the gym, cross-country skiing and 
%travelling (36 countries). I recently completed a Winter Skills course in the Cairngorm Mountains (Scotland).
%	\item I hold a full clean UK driving licence.
%	\end{newparagraph}	     
%\end{itemize}
%\end{resumeSubSectionBody}	

%*****************************************************************************%
%\resumeSection{References}
%\begin{resumeSubSectionBody}

%References can be supplied upon request.

%\textbf{Professor Simon Tett}, \tab{\textbf{Professor Mathew Williams}},\\ 
%Professor of Earth System Dynamics, \tab{Chair of Global Change Ecology},\\
%University of Edinburgh, \tab{Head of Global Change Research Institute},\\
%School of Geosciences, \tab{University of Edinburgh},\\
%Crew Building, \tab{School of GeoSciences},\\
%Alexander Crum Brown Road, \tab{Crew Building},\\   
%Edinburgh, \tab{Alexander Crum Brown Road},\\ 
%EH9 3FF. \tab{Edinburgh},\\
%e-mail: simon.tett@ed.ac.uk \tab{EH9 3FF}.\\ 
%\tab{\tab{e-mail: mat.williams@ed.ac.uk}}\\

%\textbf{Professor J. Ray Bates},\\
%Adjunct Professor of Meteorology,\\
%Meteorology and Climate Centre,\\
%School of Mathematical Sciences,\\ 
%University College Dublin,\\ 
%%Belfield,\\ 
%Dublin 4,\\
%Ireland.\\
%e-mail: ray.bates@ucd.ie

%\end{resumeSubSectionBody}

%*****************************************************************************%

\end{document}
